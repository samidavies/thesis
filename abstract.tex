This document will discuss three problems that I worked on during my Ph.D. 
Chapter \ref{chapter: SC} contains my work on the Santa Claus problem, and
Chapters \ref{chapter: S1} and \ref{chapter: S2} contain my work on scheduling with precedence constraints and communication delays.
New algorithms for scheduling and resource allocation problems have far reaching implications, as problems in scheduling and resource allocation
are a foundational playground for studying computational hardness and are practically relevant.
% We see these implications in avenues that increase the productivity of businesses, 
% like dynamically allocating to the cloud; that speed up processes in computing, 
% like training deep neural networks; and that promote social equity, 
% like helping Boston public schools design so- cioeconomic conscious bus routes [MPST18, NHP+19, BDE+20]. 

In the \emph{Santa Claus problem}, Santa has a set of gifts, and he wants to distribute them among a set of children
so that the least happy child is as happy as possible. 
Child $i$ has value $p_{ij}$ for present $j$, where $p_{ij}$ is in $ \{ 0,p_j\}$.
A modification of Haxell's hypergraph matching argument by Annamalai, Kalaitzis, and Svensson gives a $12.33$-approximation algorithm for the problem.
In joint work with Thomas Rothvoss and Yihao Zhang, we introduce a matroid version of the Santa Claus problem. 
While our algorithm is also based on the augmenting tree by Haxell, the introduction
of the matroid structure allows us to solve a more general problem with cleaner methods.
Using our result from the matroid version of the problem, we obtain a $(4+\varepsilon)$-approximation algorithm for Santa Claus.

In scheduling theory, one of the most poorly understood, yet practically interesting, models is scheduling
in the presence of \emph{communication delays}.
Here,
if two jobs are dependent and scheduled on different machines,
then at least $c$ units of time must pass between their executions.
Even for the special case where an unlimited number of identical machines are available, the best known approximation ratio
for minimizing makespan is $O(c)$.
An outstanding open problem in the top-10 list by Schuurman and Woeginger (and its recent update by Bansal)
asks whether there exists a constant-factor approximation algorithm.
In joint work with Janardhan Kulkarni, Thomas Rothvoss, Jakub Tarnawski, and Yihao Zhang,
 we prove a $O(\log c \cdot \log m)$-approximation algorithm
for the problem of minimizing makespan
on $m$ identical machines; this work is presented in Chapter \ref{chapter: S1}.
Our approach is based on a Sherali-Adams lift of a linear programming relaxation
and a randomized clustering of the semimetric space induced by this lift. 
We extend our work to the related machines setting and study the objectives of minimizing makespan 
and minimizing the weighted sum of completion times. 
Here, we also obtain polylogarithmic approximation algorithms, and these results are presented in Chapter \ref{chapter: S2}.